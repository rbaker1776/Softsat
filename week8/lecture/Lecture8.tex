\documentclass{article}

\newcommand{\titleone}{Introduction and Setup}
\newcommand{\titletwo}{C++ Programming Basics}
\newcommand{\titlethree}{How C++ Works}
\newcommand{\titlefour}{Introduction to OOP}
\newcommand{\titlefive}{Advanced OOP}
\newcommand{\titlesix}{Templates}
\newcommand{\titleseven}{The C++ Standard Library}
\newcommand{\titleeight}{Safety in C++}
\newcommand{\titlenine}{Compile-Time Programming}

\newcommand{\thistitle}{\titleeight}
\newcommand{\me}{Ryan Baker}

\usepackage{hyperref}
\usepackage[dvipsnames]{xcolor}
\usepackage{listings}

\title{\thistitle}
\author{\me}
\date{\today}

\definecolor{JadeGreen}{RGB}{91,158,62}
\definecolor{LightGray}{RGB}{247,247,247}
\lstdefinestyle{catppuccin}{
	backgroundcolor=\color{LightGray},
    commentstyle=\color{CadetBlue},
	numberstyle=\footnotesize\ttfamily\color{Gray},
	stringstyle=\color{JadeGreen},
	keywordstyle=\color{BurntOrange},
	basicstyle=\ttfamily\color{Black},
	breakatwhitespace=true,
	breaklines=true,
	captionpos=b,
	keepspaces=true,
	numbers=left,
	numbersep=5pt,
	showspaces=false,
	showstringspaces=false,
	showtabs=false,
	tabsize=4,
    frame=single,
    xleftmargin=10pt,
    xrightmargin=10pt,
    aboveskip=10pt,
    belowskip=10pt,
    framexleftmargin=10pt,
    framexrightmargin=10pt,
    framesep=5pt,
    rulecolor=\color{Gray},
}

\lstset{style=catppuccin}

\hypersetup{
	colorlinks=true,
	hidelinks=false,
	linkcolor=RoyalBlue,
	citecolor=ForestGreen,
	filecolor=DarkOrchid,
	urlcolor=BurntOrange,
	runcolor=BrickRed,
    pdftitle={\thistitle},
	pdfauthor={\me},
}

\newcommand{\inlinecpp}[1]{\lstinline[language=C++]|#1|}
\newcommand{\centercpp}[1]{\begin{center}\lstinline[language=C++]|#1|\end{center}}
\newcommand{\inputcpp}[1]{\lstinputlisting[language=C++]{#1}}


\title{\thistitle}
\author{\me}
\date{\today}

\begin{document}

\maketitle
\tableofcontents
\pagebreak

\section{Introduction to Templates}

\noindent
A \inlinecpp{template} is a \textit{very} powerful tool in C++. The basic idea is to use datatypes as parameters and have the compiler generate the relevant code for us. For example, you may want to write a function \inlinecpp{sort()} that works for different datatypes. Rather than writing and maintaining multiple \inlinecpp{sort()} functions, we can write a single \inlinecpp{sort() template} and pass the datatype as a parameter.

\subsection{How Do Templates Work?}

\noindent
Templates are expanded at compile time similar to macros. The difference is that the compiler does type checking before template expansion.

\section{Function Templates}

\noindent


\subsection{Implicit Template Deduction}

\subsection{Template Function Overloading}

\subsection{Function Template Specialization}

\section{Class Templates}

\subsection{Template Instantiation}

\subsection{Class Template Specialization}

\section{Non-Type Template Parameters}

\section{Variadic Templates}

\end{document}
