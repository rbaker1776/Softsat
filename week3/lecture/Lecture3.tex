\documentclass{article}

\newcommand{\titleone}{Introduction and Setup}
\newcommand{\titletwo}{C++ Programming Basics}
\newcommand{\titlethree}{How C++ Works}
\newcommand{\titlefour}{Introduction to OOP}
\newcommand{\titlefive}{Advanced OOP}
\newcommand{\titlesix}{Templates}
\newcommand{\titleseven}{The C++ Standard Library}
\newcommand{\titleeight}{Safety in C++}
\newcommand{\titlenine}{Compile-Time Programming}

\newcommand{\thistitle}{\titlethree}
\newcommand{\me}{Ryan Baker}

\usepackage{hyperref}
\usepackage[dvipsnames]{xcolor}
\usepackage{listings}

\title{\thistitle}
\author{\me}
\date{\today}

\definecolor{JadeGreen}{RGB}{91,158,62}
\lstdefinestyle{catppuccin}{
	backgroundcolor=\color{White},
    commentstyle=\color{CadetBlue},
	numberstyle=\footnotesize\ttfamily\color{Gray},
	stringstyle=\color{JadeGreen},
	keywordstyle=\color{BurntOrange},
	basicstyle=\ttfamily\color{Black},
	breakatwhitespace=true,
	breaklines=true,
	captionpos=b,
	keepspaces=true,
	numbers=left,
	numbersep=5pt,
	showspaces=false,
	showstringspaces=false,
	showtabs=false,
	tabsize=4,
}

\lstset{style=catppuccin}

\hypersetup{
	colorlinks=true,
	hidelinks=false,
	linkcolor=RoyalBlue,
	citecolor=ForestGreen,
	filecolor=DarkOrchid,
	urlcolor=BurntOrange,
	runcolor=BrickRed,
    pdftitle={\thistitle},
	pdfauthor={\me},
}

\newcommand{\inlinecpp}[1]{\lstinline[language=C++]|#1|}
\newcommand{\centercpp}[1]{\begin{center}\lstinline[language=C++]|#1|\end{center}}
\newcommand{\inputcpp}[1]{\lstinputlisting[language=C++]{#1}}


\title{\thistitle}
\author{\me}
\date{\today}

\begin{document}

\maketitle
\tableofcontents
\pagebreak

\section{Functions}

\noindent
Often, we want to simplify complex problems and repeat computations, and functions help by breaking tasks into smaller units. This makes the code easier to reuse, maintain, and test.

\begin{itemize}
	\item \textbf{Definition:} A function in C++ is a named block of code that takes inputs, does something with them, and can return a result.
	\item \textbf{Declaring a Function:} \inlinecpp{type name(args) \{...\}}
	\begin{itemize}
		\item[\texttt{type}:] The return type of the function (e.g., \inlinecpp{int}, \inlinecpp{void}, etc.)
		\item[\texttt{name}:] The function's name
		\item[\texttt{args}:] The function arguments or parameters as a comma separated list
	\end{itemize}
	\centercpp{int add(int a, int b, int c);}
	\item \inlinecpp{void} is used as the return type for functions that don't return anything:
	\centercpp{void greet();}
	\item \textbf{Functions and the Stack:} Every time a function is called, its local variables and arguments are pushed atop the stack.
\end{itemize}

\subsection{Passing by Value vs. Passing by Reference}

\noindent
Arguments that are passed into a function are, by default, copied. This means that the original value will remain unmodified:

\begin{lstlisting}[language=C++]
#include <iostream>

void increment(int x) { x++; }

int main()
{
	int x = 0;
	increment(x); // copy of x is passed to increment
	std::cout << x << std::endl; // x = 0
}
\end{lstlisting}

\noindent
Very often, this is the intended effect. One other consideration is that for very large objects copying is expensive. The alternative is to pass arguments by reference:

\begin{lstlisting}[language=C++]
#include <iostream>

void increment(int& x) { x++; }

int main()
{
	int x = 0;
	increment(x); // reference to x is passed
	std::cout << x << std::endl; // x = 1
}
\end{lstlisting}

\subsection{Function Overloading}

\noindent
Together, a function's name and arguments make up its \textit{signature}. Because the arguments are included, two functions with the same name but different arguments will not cause duplicate symbol errors:

\begin{lstlisting}[language=C++]
// two functions can have the same name
int add(int a, int b)        { return a + b; }
int add(int a, int b, int c) { return a + b + c; }
\end{lstlisting}

\noindent
Note that it is not enough for the functions to have differently names arguments.

\section{Scope}

\noindent
Scope is defined by \inlinecpp{\{\}} in C++. A scope is a region of the program where a varibale or symbol is valid.

\subsection{Types of Scope}

\subsubsection{Global Scope}

Variables in global scope are accessible throughout the program.

\subsubsection{Local Scope}

Variables in local, or function scope, are accessible only in their function.

\subsubsection{Anonymous Scope}

Scopes can be defined anonymously, helping with naming conflicts:

\begin{lstlisting}[language=C++]
int main()
{
	{ // anonymous scope
 		int x = 0;
	}
}
\end{lstlisting}

\subsection{Namespaces}

\noindent
Namespaces are a tool that helps us deal with naming conflicts. Consider you're writing a library with a function called \inlinecpp{init()}. You'd be bound to end up with a naming conflict, which is where namespaces help.

\begin{lstlisting}[language=C++]
namespace Library
{
	void init();
}
\end{lstlisting}

\subsubsection{Namespace Operator \inlinecpp{::}}

\centercpp{Library::init();}

\subsubsection{\inlinecpp{using} Namespaces}

\centercpp{using namespace std;}

\centercpp{using std::cout;}

\section{Conditions and Branching}

\noindent
Often times in programming, we want to choose which path to execute based on certain conditions.

\subsection{Boolean Expressions}

\noindent
A boolean variable can represent true or false. Thus, a boolean expression is an expression that evaluates to true or false.

\subsubsection{\inlinecpp{bool()} Casts}

\noindent
Evaluates to true if not 0, false if 0 for numeric types.

\subsubsection{Comparison Operators: \inlinecpp{==}, \inlinecpp{!=}, \inlinecpp{<}, \inlinecpp{<=}, \inlinecpp{>}, \inlinecpp{>=}}

\centercpp{==, !=, <, <=, >, >=}

\subsubsection{Logical Operators: \inlinecpp{!}, \inlinecpp{&&}, \texttt{||}}

\centercpp{!, &&, \|\|}

\subsection{\inlinecpp{if} Statements}

\noindent
\inlinecpp{if} Statements are used to execute a block of code if a condition is true:

\subsubsection{The Overhead of \inlinecpp{if} Statements}

After the evaluation of an if statement, the execution jumps around and loads new instructions.

\subsection{\inlinecpp{switch} Statements}

Like if statements but for integer expressions not boolean.

\subsection{Ternary Operator \inlinecpp{? :}}

\section{Loops}

\subsection{\inlinecpp{while} Loops}

\subsubsection{\inlinecpp{do while} Loops}

\subsection{\inlinecpp{for} Loops}

\subsubsection{Blank Fields}

\section{Control Flow Keywords}

\subsection{\inlinecpp{break} Keyword}

\subsection{\inlinecpp{continue} Keyword}

\subsection{\inlinecpp{return} Keyword}

\end{document}
