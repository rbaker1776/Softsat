\documentclass{article}

\usepackage{hyperref}
\usepackage[dvipsnames]{xcolor}
\usepackage{listings}
\usepackage{amsmath}

\title{All About Memory
}
\author{Ryan Baker}
\date{\today}

\definecolor{JadeGreen}{RGB}{91,158,62}
\lstdefinestyle{catppuccin}{
	backgroundcolor=\color{White},
	commentstyle=\color{Gray},
	numberstyle=\footnotesize\ttfamily\color{Gray},
	stringstyle=\color{JadeGreen},
	keywordstyle=\color{BurntOrange},
	basicstyle=\ttfamily\footnotesize\color{Black},
	breakatwhitespace=false,
	breaklines=true,
	captionpos=b,
	keepspaces=true,
	numbers=left,
	numbersep=5pt,
	showspaces=false,
	showstringspaces=false,
	showtabs=false,
	tabsize=4,
}
\lstset{style=catppuccin}

\hypersetup{
	colorlinks=true,
	hidelinks=false,
	linkcolor=RoyalBlue,
	citecolor=ForestGreen,
	filecolor=DarkOrchid,
	urlcolor=BurntOrange,
	runcolor=BrickRed,
	pdftitle={C++: From Code to Execution},
	pdfauthor={Ryan Baker},
}

\begin{document}

\maketitle
\tableofcontents
\pagebreak

\subsection*{Lecture Objectives}

\noindent
By the end of this lecture, you should:
\begin{itemize}
	\item Understand the developer tools needed to write C++
	\item Understand the build process: preprocessing, compilation, linking
	\item Be able to utilize types, arithmetic, and I/O to write a basic C++ program
\end{itemize}

\section{Introduction to C++}

\begin{itemize}
	\item What is C++?
	\begin{itemize}
		\item General-purpose programming language
		\item \textbf{Brief history:} Development, standardization, updates
		\item \textbf{Use cases:} embedded systems, HFT, OS, etc.
	\end{itemize}
	\item Why learn C++?
	\begin{itemize}
		\item There has been a concerted effort to push C++ to the side \begin{itemize}
			\item Some valid concerns (some invalid)
			\item We will investigate throughout the course
		\end{itemize}
		\item Performance, flexibility, power (sharpest tool in the toolbox)
		\item C++ has a very strong ``knowledge passport'' \begin{itemize}
			\item Becoming proficient at C++ is not easy
			\item Proficiency in C++ translates very well to other languages
		\end{itemize}
	\end{itemize}
\end{itemize}

\section{C++ Developer Tools}

\noindent
Two basic tools are needed to develop C++: a \textbf{text editor} and a \textbf{compiler}. You may also use an \textbf{Integrated Development Environment (IDE)} which is a single tool that bundles both.

\subsection{Text Editor}

\begin{itemize}
	\item What is a text editor? A tool to edit text.
	\begin{itemize}
		\item Any tool that lets you edit text is technically workable
	\end{itemize}
	\item Which text editor should I use? \begin{itemize}
		\item \href{https://code.visualstudio.com}{\textbf{Visual Studio Code IDE:}} a beginner-friendly option, very popular
		\item \href{https://www.jetbrains.com/clion/}{\textbf{CLion IDE:}} Premium IDE made by JetBrains, widely appreciated
		\item \textbf{Vim/Nvim:} My choice, steep learning curve, more rewarding \begin{itemize}
			\item Not an IDE, download instructions depend on your OS
			\item Look up ``vim install'' or ``neovim install''
			\item I recommend spending some time configuring your setup, feel free to talk to me for recommendations
		\end{itemize}
	\end{itemize}
\end{itemize}

\subsection{Compiler}

\begin{itemize}
	\item What is a compiler? \begin{itemize}
		\item A compiler is a tool that turns your code into an executable program
		\item Performs lexical analysis, syntax checking, and optimization
	\end{itemize}
	\item Which compiler should I use?
	\begin{itemize}
		\item \textbf{gcc/g++:} One of the best. Download depends on OS
		\item \textbf{clang/llvm:} My personal choice. Download depends on OS
		\begin{itemize}
			\item Best for Mac
		\end{itemize}
		\item \textbf{MSVC:} Microsoft visual compiler, a tier below \textbf{clang} and \textbf{gcc} (IMO)
		\begin{itemize}
			\item An easy option if you're developing on windows
		\end{itemize}
	\end{itemize}
\end{itemize}

\subsection{Hello, World!}

\noindent
With your text editor of choice, write the following code:

\vspace{.5em}
\lstinputlisting[language=C++]{code/helloworld.cpp}
\vspace{.5em}

\noindent
Compile your code using your compiler of choice:

\begin{itemize}
	\item[\texttt{>>}] \texttt{g++ helloworld.cpp -o exe}
	\item[\texttt{>>}] \texttt{clang++ helloworld.cpp -o exe}
\end{itemize}

\noindent
These commands will compile \texttt{helloworld.cpp} and output an executable \texttt{exe}. You can run the executable file:

\begin{verbatim}
	>> ./exe
	Hello, World!
\end{verbatim}

\noindent
The remainder of this lecture will be spent on understanding the process that transforms \texttt{helloworld.cpp} into an executable that prints ``Hello, World!''.

\section{The C++ Build Process}

\noindent
How do we get from the \texttt{.cpp} file to an executable program? Answering this question will allow us to understand and debug our programs more efficiently.

\subsection{Source Code}

\begin{itemize}
	\item \textbf{Source code:} Human-readable code written in \texttt{.cpp} files
	\item $\texttt{main.cpp}\rightarrow\boxed{\textrm{Preprocessor}}\rightarrow\boxed{\textrm{Compiler}}\rightarrow\boxed{\textrm{Linker}}\rightarrow\texttt{main.exe}$
	\item \textbf{Key point:} Source code is just a text file: \begin{itemize}
		\item[\texttt{>>}] \texttt{cp helloworld.cpp ryan.baker}
		\item[\texttt{>>}] \texttt{clang++ -x c++ ryan.baker -o exe \&\& ./exe}
		\begin{itemize}
			\item There is nothing special about \texttt{.cpp} files. The compiler just needs text written in a language called C++.
			\item \texttt{-x} is a flag that clang takes that says: ``The language is C++, even if the file extensions are weird''
		\end{itemize}
	\end{itemize}
\end{itemize}

\subsection{Preprocessor}

\begin{itemize}
	\item What does the preprocessor do? \begin{itemize}
		\item Text substitution: \texttt{\#define MAX 500} replaces ``\texttt{MAX}'' with ``\texttt{500}''
		\item File inclusion: \texttt{\#include <iostream>} copies and pastes \texttt{iostream}
		\begin{itemize}
			\item \texttt{\#include "filename"}: searches current folder, use for local files
			\item \texttt{\#include <filename>}: searches standard include directories
		\end{itemize}
		\item Conditional compilation: \texttt{\#if} and others to select code that compiles
		\end{itemize}
		\vspace{.5em}
		\lstinputlisting[language=C++]{code/preprocessor.cpp}
	\item \textit{Probably won't discuss} \texttt{\#pragma}
	\begin{itemize}
		\item Less relevant in C++ than in C, usually bad practice
	\end{itemize}
	\item Use of \texttt{-E} to view preprocessor output \begin{itemize}
		\item[\texttt{>>}] \texttt{clang++ -E preprocessor.cpp > preprocessor\_output.cpp} \begin{itemize}
			\item \texttt{iostream} file has been copied and pasted in
			\item macros \texttt{MAX} and \texttt{STATUS} have been expanded to their values
		\end{itemize}
	\end{itemize}
	\item \textbf{Key point:} \texttt{preprocessor\_output.cpp} is the same as \texttt{preprocessor.cpp} \begin{itemize}
		\item[\texttt{>>}] \texttt{clang++ preprocessor\_output.cpp -o exe \&\& ./exe}
		\begin{itemize}
			\item This command produces the same thing as the first program
		\end{itemize}
	\end{itemize}
\end{itemize}

\subsection{Compiler}

\begin{itemize}
	\item What does the compiler do? \begin{itemize}
		\item Reads the output of the preprocessor and turns it to machine code
		\item Responsible for alerting the user about various types of errors
		\item Outputs object files (\texttt{.o} or \texttt{.obj}) \begin{itemize}
			\item Mostly machine code, has some directives for the linker to use
		\end{itemize}
	\end{itemize}
	\item Use of \texttt{-c} to view the object file (completely unreadable)\begin{itemize}
		\item \texttt{-c} stands for ``just compile'' (don't link)
		\item[\texttt{>>}] \texttt{clang++ -c helloworld.cpp} produces \texttt{helloworld.o}
		\item[\texttt{>>}] \texttt{clang++ helloworld.o -o exe \&\& ./exe} runs the program
	\end{itemize}
	\item Use of \texttt{-S} to view the assembly output (readable)\begin{itemize}
		\item[\texttt{>>}] \texttt{clang++ -S helloworld.cpp} produces \texttt{helloworld.s}
		\item Contains traces of our original program
		\item[\texttt{>>}] \texttt{clang++ helloworld.s -o exe \&\& ./exe} runs the program
	\end{itemize}
\end{itemize}

\subsection{Linker}

\begin{itemize}
	\item What does the linker do? \begin{itemize}
		\item Resolves symbols, matching declarations to definitions
		\item Combines multiple translation units into an executable
		\item This allows us to write code across files, enhancing modularity
	\end{itemize}
	
	\lstinputlisting[language=C++]{code/file2.hpp}
	\lstinputlisting[language=C++]{code/file1.cpp}
	\lstinputlisting[language=C++]{code/file2.cpp}

	\item \textbf{Key point:} \texttt{clang++ file1.cpp -o exe} produces a linker error \begin{itemize}
		\item The symbol ``\texttt{greet()}'' is not defined
		\item Recognize the difference between compilation and linking errors \begin{itemize}
			\item Linker errors are often more convoluted
			\item Often denoted by ``\texttt{ld:}'' or ``\texttt{linker error:}'' 
		\end{itemize}
	\end{itemize}
	\item[\texttt{>>}] \texttt{clang++ file1.cpp file2.cpp -o exe} \begin{itemize}
		\item We need to pass in \texttt{file2.cpp} to the linker
	\end{itemize}
\end{itemize}

\section{C++ Programming Basics}

\subsection{Types and Variables}

\noindent
All programming involves storing and manipulating data, typically in variables. A variable's \textit{datatype} defines the set of values it can hold. For example, a \texttt{char}acter datatype represents letters like `a' through `z', while a \texttt{bool}ean datatype represents true or false.

\vspace{1em}	
\noindent
C++ has some built-in datatypes, called \textit{primitive} or \textit{integral} datatypes. Each datatype is designed to serve a different purpose. However, with a low-level programming language such as C++, \textbf{the only real difference} between any of these datatypes is \textbf{the amount of memory} they occupy.

\begin{itemize}
	\item \texttt{int}: represents an integer
		\begin{itemize}
			\item[\texttt{>>}] \texttt{int x = 42;} assigns the value 42 to a variable `x'
			\item[\texttt{>>}] \texttt{sizeof(x); sizeof(int)} returns the \# of bytes `x' occupies
			\item Because `x' occupies a finite number of bytes, its range is limited
			\begin{itemize}
				\item We can calculate its total range as $2^w$ where $w$ represents the width of `x' in bits
				\item Note that the maximum value may be halved if `x' is signed
			\end{itemize}
			\item If we want a smaller int, we use \texttt{short}. If we want a longer int, we can use \texttt{long} or \texttt{long long}
			\begin{itemize}
				\item I find these names very confusing
				\item I recommend \texttt{\#include <cstdint>}
			\end{itemize}
			\item Signedness: we can prepend a `u` or \texttt{unsigned} to the type to make the number unsigned. This expands its positive range
		\end{itemize}
		\item \texttt{char}: represents a character
		\begin{itemize}
			\item[\texttt{>>}] \texttt{char letter = `a';} assigns `a' to `letter'
				\begin{itemize}
					\item \texttt{`a'} is really just a number
					\item[\texttt{>>}] \texttt{int x = `a'; std::cout << x << std::endl;}
				\end{itemize}
		\end{itemize}
		\item \texttt{float}, \texttt{double}: represents a floating-point (fractional) type
		\begin{itemize}
			\item \texttt{double} is (usually) twice as large as a \texttt{float}
			\item \texttt{sizeof(double)} = 8, \texttt{sizeof(float)} = 4 (usually)
		\end{itemize}
		\item \texttt{bool}: represents a boolean value (True or False)
		\begin{itemize}
			\item \texttt{sizeof(bool)} = 1 (usually)
		\end{itemize}
		\item \texttt{void}: represents ``no type''
		\begin{itemize}
			\item \texttt{sizeof(void)} is a senseless operation, produces an error
	\end{itemize}
\end{itemize}

\subsection{Input and Output with \texttt{iostream}}

\noindent
Our programs are useless unless we can communicate with them. C++ provides various methods of passing data into and out-of our programs. The \texttt{iostream} library is the most widely used library for input and printing data in C++.

\begin{itemize}
	\item ``\texttt{iostream}'' stands for input and output (IO) stream
	\begin{itemize}
		\item[\texttt{>>}] \texttt{\#include <iostream>}
	\end{itemize}
	\item Output with \texttt{iostream}
	\begin{itemize}
		\item \texttt{std::cout} is used to print data to the console
		\begin{itemize}
			\item \texttt{std::} is a namespace access, says search namespace ``\texttt{std}'' (standard) for function called ``\texttt{cout}''
			\item We will discuss namespaces later in the course
		\end{itemize}
		\item \texttt{std::endl} is used to output a newline and flush the buffer
		\item The `\texttt{<<}' operator is the output stream operator
	\end{itemize}
	\item Input with \texttt{iostream}
	\begin{itemize}
		\item \texttt{std::cin} is used to fetch data at runtime from the user
		\item The `\texttt{>>}' operator is the stream input operator
	\end{itemize}
\end{itemize}

\noindent
We will discuss the workings of \texttt{iostream} more in depth when we discuss \textbf{streams}. Presently, just get familiar with the syntax of \texttt{cin} and \texttt{cout}.

\vspace{.5em}
\lstinputlisting[language=C++]{code/iostreamio.cpp}

\end{document}